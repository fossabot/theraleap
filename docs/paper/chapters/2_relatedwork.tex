\chapter{Related Work}
\label{sec:relatedwork}

The general feasibility of using devices such as the Leap Motion for different usecases than their primary intended purpose in the entertainment industry, specifically using them for clinical purposes, has already been shown by various research.

In 2014, researchers from the University of California have successfully developed an Exergame version of the popular smartphone game \emph{Fruit Ninja}, utilizing the Leap Motion Device for domain virtualization \cite{khademi2014free}. The researches have shown that the Game could purposefully be used for stroke rehabilitation by showing that a strong correlation exists between the achieved score while playing the \emph{Fruit Ninja} game, and standard clinical assessment scores.

The feasibility of using the Leap Motion device for developing Exergames specifically in the context of the Web browser have been shown by \cite{DigitizingHandRehabilitation}. In the work, the authors showed the general feasibility of web based digital hand rehabilitation by implementing and evaluating Web based Exergames on a prototypical level. In an earlier work, the same researchers have shown that the Leap Motion device is capable of delivering a virtualized hand representation of sufficient accuracy for tracking the patients rehabilitation progress \cite{AnalyzingHandTherapySuccess}.

In a research project completed in 2017, students of the DHBW Stuttgart initially collaborated with ergo therapists of the Katharinenhostpital Stuttgart in order to determine requirements for a web based handtherapy system that would be mature enough to be deployed at hospitals at a bigger scale \cite{StudiArbeitVolzBaumotte}. In the project, some preliminary system requirements have been identified, and recommendations for the architecture and implementation of the system have been given. In addition, some of the recovery exercises that the system should be able to assimilate have been documented.
\\\\
This work follows up on the previous research by providing an architecture and reference implementation for a mature, holistic hand therapy system that is ready to be used by ergo therapists and their patients as an accompaniment for hand and wrist injury rehabilitation.