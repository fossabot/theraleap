\chapter{Conclusion}
\label{sec:conclusion}
The work presented in this paper confirms that the modern web is mature and performant enough in order to fulfill all imposed requirements on the computer aided ergo therapy system. The provided implementation is complete to the degree that the system is ready to be integrated into recovery sessions. In addition, a pilot phase with motivated patients could be performed in order to collect initial end user feedback. However, in order for the full vision of the patient doing a large portion of the recovery exercises at home, and the therapist being able to closely monitor progress to become a reality, a lot more work is yet to be done. The next logical implementation steps have been detailed in chapter \ref{sec:future}. The development time required until the System can be considered complete from an architectural perspective will probably fill the pages of at least one additional Student Research Project, not even considering the design and implementation of Preprocessors, Exercise Classifiers, and Games.

I hope that others might find this work useful for their purposes, and maybe even advance its development in order for it to potentially be considered an important component in the recovery period after hand injuries in the future. In order for this to be possible, I am releasing all materials generated while working on this project under an Open Source License, thus making it accessible to everyone.