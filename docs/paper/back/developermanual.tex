\chapter{Developer Manual}
Developers are advised to read through the main part of the paper in order to gain a general understanding on how the framework functions. Additionally, instructions on how to compile the project, and run the project in a development mode are given in the \texttt{README.me}, in the Source Code Repository (see Appendix \ref{app:code}). This Manual briefly gives practical advice on how to implement the System Interfaces, namely the \texttt{Preprocessing}, \texttt{Gesture Classification}, and \texttt{Game} Interfaces.
\section{Adding a Preprocessor}
As an example, let's add a Preprocessor for the Leap Motion Device that filters out frames where more than one hand is detected.
\subsection{Implement the Preprocessor}
This Preprocessor is specific to Leap Motion Device data, so be sure to put it into the \texttt{src/processing/leap} folder.
\begin{minted}{typescript}
// src/processing/leap/droptwohands.ts
export const DropTwoHandsId = "DropTwoHands";

export class DropTwoHandsPreProcessor
  implements Operator<GenericHandTrackingData, GenericHandTrackingData> {
  constructor() {}

  public call(
    subscriber: Subscriber<GenericHandTrackingData>,
    source: Observable<GenericHandTrackingData>
  ) {
    return source
      .pipe(
        // filter every frame where the hand
        // array is longer than 1.
        filter((value: LeapHandTrackingData) => value.data.hands.length > 1)
      )
      .subscribe(subscriber);
  }
}
\end{minted}

\subsection{Register the Preprocessor to the Framework}
in \texttt{src/processing/resolver.ts}, register your Preprocessor.
\begin{minted}{typescript}
// src/processing/resolver.ts
export const ResolverRegistry: {
  [_: string]: { new (...args: any[]): Operator<any, any> };
} = {
  [DropNFramesPreProcessorId]: DropNFramesOperator,
  [DestroyUselessFramesId]: DestroyUselessFramesOperator,
  [FPSThrottlerId]: FPSThrottler,
  // our new preprocessor
  [DropTwoHandsId]; DropTwoHandsPreProcessor
};
\end{minted}

\subsection{Add the new Preprocessor to the Vuex State}
To the \texttt{preprocessors.ts} state submodule, add a the state of the preprocessor, and implement the \texttt{constructConfig} method.
\begin{minted}{typescript}
// src/state/modules/preprocessors.ts
state: {
    preprocesors: {
        ...
        dropTwoHandsPreProcessor: {
            enabled: false,
            constructConfig: () => {
                identifier: DropTwoHandsId,
                // We don't need args for this
                args: []
            }
        }
        ...
    }
}
\end{minted}
\subsection{Add a Description to the Frontend}
Create a new Vue Component, containing an \texttt{md-card}, describing your Preprocessor. Import and integrate it in the \texttt{PreProcessing.vue} Frontend Component. If the switch is flipped by the User, call \texttt{modifyPreProcessor}, and commit a state mutation to set the \texttt{enabled} property of your preprocessor to \texttt{true}. Finally, call \texttt{preprocessorSelectionUpdated}. This will tell the Device Driver that the Preprocessors have changed, call the state function \text{constructConfig} internally, and construct your Preprocessor, supplying any arguments to the constructor in the order you specified in the \texttt{args} array.
\section{Adding a Classifier}
Let's add a classifier that randomly classifies every nth frame as a Gesture.
\subsection{Implement the Classifier}
\begin{minted}{typescript}
// src/classify/classifiers/randomtestclassifier.ts
export const RandomTestClassifierId = "RandomTestClassifier";
export class RandomTestClassifier
  implements Operator<LeapHandTrackingData, ClassificationData> {
  constructor(
    // With this parameter we'll configure on which
    // frame we'll emit the Classification
    private n: number,
  ) {}

  public call(
    subscriber: Subscriber<ClassificationData>,
    source: Observable<LeapHandTrackingData>
  ) {
    return source
      .pipe(
          // Buffer n Frames
          bufferCount(this.n),
          // Transform the result into ClassificationData
          map((_ : LeapHandTrackingData[]) => {
              return {
                actionName: "ONE_SHOT",
                metrics: {
                    cheatFactor: 0,
                    quality: 0
                }
              }
          })
      )
      .subscribe(subscriber);
  }
}
\end{minted}
\subsection{Register the Classifier to the Framework}
In \texttt{src/classify/resolver.ts}, register the Classifier so the Resolver knows how to construct an instance of it.
\begin{minted}{typescript}
export const ClassifierRegistry: {
  [_: string]: { new (...args: any[]): Operator<any, any> };
} = {
  [ThumbSpreadClassifierId]: ThumbSpreadClassifier,
  // our shiny new classifier
  [RandomTestClassifierId]: RandomTestClassifier
};
\end{minted}
\subsection{Add the Classifier to the VueX state}
Next, let's model the State of the Classifier Frontend Component. In \texttt{classifiers.ts}:

\begin{minted}{typescript}
// src/state/modules/classifiers.ts
state: {
classifiers: {
    ...
    RandomTestClassifier: {
        enabled: false,
        n: 150,
        constructConfig: () => {
            return {
                identifier: RandomTestClassifierId,
                args: [
                    classifier.state.classifiers.RandomTestClassifier.n
                ]
            }
        }
    },
    ...
}
\end{minted}
\subsection{Add a Description to the Frontend}
Create a new Vue Component, containing an \texttt{md-card}, describing your Classifier, how it works, and how to configure it. Import and integrate it in the \texttt{Classifiers.vue} Frontend Component. If the switch is flipped by the User, or any Configuration property has changed call \texttt{classifierSelectionUpdated}.

\section{Adding a Game}
\subsection{Implement the Game}
If your Game needs any third party assets, such as images, put them into the \texttt{src/dist} directory, or alternatively import them using Webpack. 

To add a Game, add a class implementing the \texttt{Game} interface, and make it the default export. In the class, you can import any third party libraries you may need. All of these will be lazy loaded by the Game Execution Framework, so you don't need to worry about making a negative impact on bundle size.

\subsection{Register the Game to the Framework}
Next, tell the Framework how to construct the game. The \texttt{import()} function should receive as an argument the path to the class implementing the \texttt{Game} Interface (or alternatively, the directory name. This works if there is an \texttt{index.ts} file in the directory.).
\begin{minted}{typescript}
// src/games/resolver.ts
export const GameResolveMapping: { [key: string]: () => Promise<any> } = {
  "space-shooter": () =>
    import("@/games/space-shooter").then((imp: any) => {
      return imp.default;
    }),
  // our new game
  "new-game"; () =>
    import("@/games/new-game").then((imp: any) => {
      return imp.default;
    }),
};
\end{minted}
\subsection{Add the Game to the GameList}
Finally, add the Game to the GameList component. Add a new card containing a brief description and title. Use the existing Button Components for rendering the Play Buttons. On click, call \texttt{play('new-game')} (the key you put into \texttt{GameResolveMapping}).