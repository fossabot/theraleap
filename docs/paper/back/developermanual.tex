\chapter{Developer Manual}
Developers are advised to read through the main part of the paper in order to gain a general understanding on how the framework functions. Additionally, instructions on how to compile the project, and run the project in a development mode are given in the \texttt{README.me}, in the Source Code Repository (see Appendix \ref{app:code}). This Manual briefly gives practical advice on how to implement the System Interfaces, namely the \texttt{Preprocessing}, \texttt{Gesture Classification}, and \texttt{Game} Interfaces.
\section{Adding a Preprocessor}
As an example, let's add a Preprocessor for the Leap Motion Device that filters out frames where more than one hand is detected.
\subsection{Implement the Preprocessor}
This Preprocessor is specific to Leap Motion Device data, so be sure to put it into the \texttt{src/processing/leap} folder.
\begin{minted}{typescript}
// src/processing/leap/droptwohands.ts
export const DropTwoHandsId = "DropTwoHands";

export class DropTwoHandsPreProcessor
  implements Operator<GenericHandTrackingData, GenericHandTrackingData> {
  constructor() {}

  public call(
    subscriber: Subscriber<GenericHandTrackingData>,
    source: Observable<GenericHandTrackingData>
  ) {
    return source
      .pipe(
        // filter every frame where the hand
        // array is longer than 1.
        filter((value: LeapHandTrackingData) => value.data.hands.length > 1)
      )
      .subscribe(subscriber);
  }
}
\end{minted}

\subsection{Register the Preprocessor to the Framework}
in \texttt{src/processing/resolver.ts}, register your Preprocessor.
\begin{minted}{typescript}
// src/processing/resolver.ts
export const ResolverRegistry: {
  [_: string]: { new (...args: any[]): Operator<any, any> };
} = {
  [DropNFramesPreProcessorId]: DropNFramesOperator,
  [DestroyUselessFramesId]: DestroyUselessFramesOperator,
  [FPSThrottlerId]: FPSThrottler,
  // our new preprocessor
  [DropTwoHandsId]; DropTwoHandsPreProcessor
};
\end{minted}

\subsection{Add the new Preprocessor to the Vuex State}
To the \texttt{preprocessors.ts} state submodule, add a the state of the preprocessor, and implement the \texttt{constructConfig} method.
\begin{minted}{typescript}
// src/state/modules/preprocessors.ts
state: {
    preprocesors: {
        ...
        dropTwoHandsPreProcessor: {
            enabled: false,
            constructConfig: () => {
                identifier: DropTwoHandsId,
                // We don't need args for this
                args: []
            }
        }
        ...
    }
}
\end{minted}
\subsection{Add a Description to the Frontend}
Create a new Vue Component, containing an \texttt{md-card}, describing your Preprocessor. Import and integrate it in the \texttt{PreProcessing.vue} Frontend Component. If the switch is flipped by the User, call \texttt{modifyPreProcessor}, and commit a state mutation to set the \texttt{enabled} property of your preprocessor to \texttt{true}. Finally, call \texttt{preprocessorSelectionUpdated}. This will tell the Device Driver that the Preprocessors have changed, call the state function \text{constructConfig} internally, and construct your Preprocessor, supplying any arguments to the constructor in the order you specified in the \texttt{args} array.
Create a new C
\section{Adding a Classifier}
\section{Adding a Game}